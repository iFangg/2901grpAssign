%%%%%%%%%%%%%%%%%%%%%%%%%%%%%%%%%%%%%%%%%
% Lachaise Assignment
% LaTeX Template
% Version 1.0 (26/6/2018)
%
% This template originates from:
% http://www.LaTeXTemplates.com
%
% Authors:
% Marion Lachaise & François Févotte
% Vel (vel@LaTeXTemplates.com)
%
% License:
% CC BY-NC-SA 3.0 (http://creativecommons.org/licenses/by-nc-sa/3.0/)
% 
%%%%%%%%%%%%%%%%%%%%%%%%%%%%%%%%%%%%%%%%%

%----------------------------------------------------------------------------------------
%	PACKAGES AND OTHER DOCUMENT CONFIGURATIONS
%----------------------------------------------------------------------------------------

\documentclass{article}

\usepackage{parskip}
\usepackage{amsmath}
\usepackage{amsthm}
\usepackage{mathtools}
\usepackage{pgfplots}

\DeclareMathOperator{\Ima}{im}

%%%%%%%%%%%%%%%%%%%%%%%%%%%%%%%%%%%%%%%%%
% Lachaise Assignment
% Structure Specification File
% Version 1.0 (26/6/2018)
%
% This template originates from:
% http://www.LaTeXTemplates.com
%
% Authors:
% Marion Lachaise & François Févotte
% Vel (vel@LaTeXTemplates.com)
%
% License:
% CC BY-NC-SA 3.0 (http://creativecommons.org/licenses/by-nc-sa/3.0/)
% 
%%%%%%%%%%%%%%%%%%%%%%%%%%%%%%%%%%%%%%%%%

%----------------------------------------------------------------------------------------
%	PACKAGES AND OTHER DOCUMENT CONFIGURATIONS
%----------------------------------------------------------------------------------------

\usepackage{amsmath,amsfonts,stmaryrd,amssymb} % Math packages

\usepackage{enumerate} % Custom item numbers for enumerations

\usepackage[ruled]{algorithm2e} % Algorithms

\usepackage[framemethod=tikz]{mdframed} % Allows defining custom boxed/framed environments

\usepackage{listings} % File listings, with syntax highlighting
\lstset{
	basicstyle=\ttfamily, % Typeset listings in monospace font
}

%----------------------------------------------------------------------------------------
%	DOCUMENT MARGINS
%----------------------------------------------------------------------------------------

\usepackage{geometry} % Required for adjusting page dimensions and margins

\geometry{
	paper=a4paper, % Paper size, change to letterpaper for US letter size
	top=2.5cm, % Top margin
	bottom=3cm, % Bottom margin
	left=2.5cm, % Left margin
	right=2.5cm, % Right margin
	headheight=14pt, % Header height
	footskip=1.5cm, % Space from the bottom margin to the baseline of the footer
	headsep=1.2cm, % Space from the top margin to the baseline of the header
	%showframe, % Uncomment to show how the type block is set on the page
}

%----------------------------------------------------------------------------------------
%	FONTS
%----------------------------------------------------------------------------------------

\usepackage[utf8]{inputenc} % Required for inputting international characters
\usepackage[T1]{fontenc} % Output font encoding for international characters

\usepackage{XCharter} % Use the XCharter fonts

%----------------------------------------------------------------------------------------
%	COMMAND LINE ENVIRONMENT
%----------------------------------------------------------------------------------------

% Usage:
% \begin{commandline}
%	\begin{verbatim}
%		$ ls
%		
%		Applications	Desktop	...
%	\end{verbatim}
% \end{commandline}

\mdfdefinestyle{commandline}{
	leftmargin=10pt,
	rightmargin=10pt,
	innerleftmargin=15pt,
	middlelinecolor=black!50!white,
	middlelinewidth=2pt,
	frametitlerule=false,
	backgroundcolor=black!5!white,
	frametitle={Command Line},
	frametitlefont={\normalfont\sffamily\color{white}\hspace{-1em}},
	frametitlebackgroundcolor=black!50!white,
	nobreak,
}

% Define a custom environment for command-line snapshots
\newenvironment{commandline}{
	\medskip
	\begin{mdframed}[style=commandline]
}{
	\end{mdframed}
	\medskip
}

%----------------------------------------------------------------------------------------
%	FILE CONTENTS ENVIRONMENT
%----------------------------------------------------------------------------------------

% Usage:
% \begin{file}[optional filename, defaults to "File"]
%	File contents, for example, with a listings environment
% \end{file}

\mdfdefinestyle{file}{
	innertopmargin=1.6\baselineskip,
	innerbottommargin=0.8\baselineskip,
	topline=false, bottomline=false,
	leftline=false, rightline=false,
	leftmargin=2cm,
	rightmargin=2cm,
	singleextra={%
		\draw[fill=black!10!white](P)++(0,-1.2em)rectangle(P-|O);
		\node[anchor=north west]
		at(P-|O){\ttfamily\mdfilename};
		%
		\def\l{3em}
		\draw(O-|P)++(-\l,0)--++(\l,\l)--(P)--(P-|O)--(O)--cycle;
		\draw(O-|P)++(-\l,0)--++(0,\l)--++(\l,0);
	},
	nobreak,
}

% Define a custom environment for file contents
\newenvironment{file}[1][File]{ % Set the default filename to "File"
	\medskip
	\newcommand{\mdfilename}{#1}
	\begin{mdframed}[style=file]
}{
	\end{mdframed}
	\medskip
}

%----------------------------------------------------------------------------------------
%	NUMBERED QUESTIONS ENVIRONMENT
%----------------------------------------------------------------------------------------

% Usage:
% \begin{question}[optional title]
%	Question contents
% \end{question}

\mdfdefinestyle{question}{
	innertopmargin=1.2\baselineskip,
	innerbottommargin=0.8\baselineskip,
	roundcorner=5pt,
	nobreak,
	singleextra={%
		\draw(P-|O)node[xshift=1em,anchor=west,fill=white,draw,rounded corners=5pt]{%
		Question \theQuestion\questionTitle};
	},
}

\newcounter{Question} % Stores the current question number that gets iterated with each new question

% Define a custom environment for numbered questions
\newenvironment{question}[1][\unskip]{
	\bigskip
	\stepcounter{Question}
	\newcommand{\questionTitle}{~#1}
	\begin{mdframed}[style=question]
}{
	\end{mdframed}
	\medskip
}

%----------------------------------------------------------------------------------------
%	WARNING TEXT ENVIRONMENT
%----------------------------------------------------------------------------------------

% Usage:
% \begin{warn}[optional title, defaults to "Warning:"]
%	Contents
% \end{warn}

\mdfdefinestyle{warning}{
	topline=false, bottomline=false,
	leftline=false, rightline=false,
	nobreak,
	singleextra={%
		\draw(P-|O)++(-0.5em,0)node(tmp1){};
		\draw(P-|O)++(0.5em,0)node(tmp2){};
		\fill[black,rotate around={45:(P-|O)}](tmp1)rectangle(tmp2);
		\node at(P-|O){\color{white}\scriptsize\bf !};
		\draw[very thick](P-|O)++(0,-1em)--(O);%--(O-|P);
	}
}

% Define a custom environment for warning text
\newenvironment{warn}[1][Warning:]{ % Set the default warning to "Warning:"
	\medskip
	\begin{mdframed}[style=warning]
		\noindent{\textbf{#1}}
}{
	\end{mdframed}
}

%----------------------------------------------------------------------------------------
%	INFORMATION ENVIRONMENT
%----------------------------------------------------------------------------------------

% Usage:
% \begin{info}[optional title, defaults to "Info:"]
% 	contents
% 	\end{info}

\mdfdefinestyle{info}{%
	topline=false, bottomline=false,
	leftline=false, rightline=false,
	nobreak,
	singleextra={%
		\fill[black](P-|O)circle[radius=0.4em];
		\node at(P-|O){\color{white}\scriptsize\bf i};
		\draw[very thick](P-|O)++(0,-0.8em)--(O);%--(O-|P);
	}
}

% Define a custom environment for information
\newenvironment{info}[1][Info:]{ % Set the default title to "Info:"
	\medskip
	\begin{mdframed}[style=info]
		\noindent{\textbf{#1}}
}{
	\end{mdframed}
}
 % Include the file specifying the document structure and custom commands

%----------------------------------------------------------------------------------------
%	ASSIGNMENT INFORMATION
%----------------------------------------------------------------------------------------

\title{Math2901: Group Assignment} % Title of the assignment

\author{Dylan Wang, Ivan Fang\\ \texttt{z5422214@ad.unsw.edu.au, z5418045@ad.unsw.edu.au}} % Author name and email address

\date{University of New South Wales --- \today} % University, school and/or department name(s) and a date

%----------------------------------------------------------------------------------------

\begin{document}

\maketitle % Print the title

%----------------------------------------------------------------------------------------
%	INTRODUCTION
%----------------------------------------------------------------------------------------

\section*{Question 1 Solutions} % Unnumbered section
Let $A, B\subseteq\Omega$.

1. 

Given that event $A$ is independent of itself, this means that $\mathbb{P}(A|A) = \mathbb{P}(A)$. This implies $\mathbb{P}(A) = \mathbb{P}(A)$ and so $\mathbb{P}(A)$ must equal either $1$ or $0$.

2.

Given event $A$ such that $\mathbb{P}(A) = 1$ or $\mathbb{P}(A) = 0$, we observe the conditional probability between events $A$ and $B$.

For the case where $\mathbb{P}(A) = 1$:
\begin{align*}
    \mathbb{P}(B|A) &= \frac{\mathbb{P}(B\cap A)}{\mathbb{P}(A)}\\
    &= \mathbb{P}(B\cap A)\\
    &= \mathbb{P}(B)\mbox{.}
\end{align*}
\hspace*{6mm}We reach this result because the intersection with a guaranteed event will always be the probability of the other event.

For the case where $\mathbb{P}(A) = 0$:
\begin{align*}
    \mathbb{P}(A|B) &= \frac{\mathbb{P}(A\cap B)}{\mathbb{P}(B)}\\
    &= \frac{\mathbb{P}(A) + \mathbb{P}(B) - \mathbb{P}(A\cup B)}{\mathbb{P}(B)}\\
    &= \frac{\mathbb{P}(B) - \mathbb{P}(A\cup B)}{\mathbb{P}(B)}\\
    &= \frac{\mathbb{P}(B) - \mathbb{P}(B)}{\mathbb{P}(B)}\\
    &= 0\\
    &= \mathbb{P}(A)\mbox{.}
\end{align*}
\hspace*{6mm} Since the union with an impossible event always returns the probability of the other event, we reach our result.

3.

Note, by the Total Law of Probability, $\mathbb{P}(A\cap B) \leq 1$.

Now observe,
\begin{align*}
    \mathbb{P}(A\cap B) & = \mathbb{P}(A) +\mathbb{P}(B) - \mathbb{P}(A\cap B)\\    
    1 &\geq \mathbb{P}(A) +\mathbb{P}(B) - \mathbb{P}(A\cap B)\\
    \mathbb{P}(A\cap B)\ &\geq \mathbb{P}(A) +\mathbb{P}(B) - 1\mbox{.}
\end{align*}
\hspace*{6mm} The inequality has been proven, thus we are done.

4.

Using the previous proven inequality,
\begin{align*}
    \mathbb{P}(A_1\cap A_2)\ &\geq \mathbb{P}(A_1) + \mathbb{P}(A_2) - 1\\
    \mathbb{P}(A_1\cap A_2\cap A_3)\ &\geq \mathbb{P}(A_1) + \mathbb{P}(A_2) - 1\\
    &= \mathbb{P}(A_1) +\mathbb{P}(A_2) - 1 +  \mathbb{P}(A_3) - 1\\
    &= \mathbb{P}(A_1) +\mathbb{P}(A_2) +  \mathbb{P}(A_3) - 2\\
    &\vdots\\
    \mathbb{P}(\bigcap_{i = 1}^n A_i)\ &\geq \sum_{i = 1}^{n} A_i - (n-1)\mbox{.}
\end{align*}
\hspace*{6mm} We have thus proved the inequality. 

\section*{Question 2 Solutions} % Unnumbered section
See below (page 6)
\pagebreak
\section*{Question 3 Solutions} % Unnumbered section
1.

We can see that $\tilde{X_n} = \bar{X_n} + \frac{1}{n}(2X_1)-\frac{1}{n}\left(X_{n-1}+X_n\right)$. Now calculating the bias of $\tilde{X_n}$
\begin{align*}
    \mbox{Bias}(\tilde{X_n}) &=\mbox{E}(\tilde{X_n}) - \mu\\
    &= \frac{1}{n}\mbox{E}\left(\sum_{i = 1}^{n}X_i + 2X_1 - (X_{n-1}+X_n)\right) - \mu\\
    &= \frac{1}{n}\mbox{E}\left(\sum_{i = 1}^{n}X_i\right)+\frac{1}{n}\mbox{E}(2X_1)-\frac{1}{n}\mbox{E}(X_{n-1}+X_n) - \mu\\
    &= \frac{1}{n}\mbox{E}\left(\sum_{i = 1}^{n}X_i\right)+\frac{2}{n}\mbox{E}(X_1)-\frac{1}{n}\mbox{E}(X_{n-1})-\frac{1}{n}\mbox{E}(X_n) - \mu\mbox{.}
\end{align*}
\hspace*{6mm}Because the random sample is independent identically distributed, the expected value of each variable in the sample is equal to the mean. This means that our simplified equation becomes
\begin{align*}
    \mu + \frac{2}{n}\mu -\frac{1}{n}\mu -\frac{1}{n}\mu -\mu = 0\mbox{.}
\end{align*}
We have thus shown that $\tilde{X_n}$ is an unbiased estimator of $\mu$.

2.

The mean square error can be written as
\begin{align*}
    \mbox{MSE}(\tilde{X_n}) &= \mbox{Var}(\tilde{X_n})+(\mbox{Bias}(\tilde{X_n}))^2\\
    &= \mbox{Var}(\tilde{X_n})\\
    &= \mbox{Var}\left(\frac{3X_1+\sum_{i = 2}^{n-2}X_i}{n}\right)\\
    &= \frac{1}{n^2}\left(9\mbox{Var}(X_1) + \sum_{i = 2}^{n-2}\mbox{Var}(X_i)\right)\\
    &= \frac{9\sigma^2 + \sigma^2(n - 3)}{n}\\
    &= \frac{6\sigma^2 + n\sigma^2}{n^2}\mbox{.}
\end{align*}
Thus, the MSE is $\frac{1}{n^2}(6\sigma^2 + n\sigma^2)$.

3.
\begin{align*}
    \lim_{n\rightarrow\infty}\mbox{MSE}(\tilde{X_n}) &= \lim_{n\rightarrow\infty}\frac{1}{n^2}(6\sigma^2 + n\sigma^2)\\
    &= 0\mbox{.}
\end{align*}

4.

To measure the better estimate of $\mu$, we can observe the variances of each
\begin{equation*}
    \mbox{Var}(\bar{X_n}) = \frac{\sigma^2}{n}\leq \frac{1}{n^2}(6\sigma^2 + n\sigma^2) = \mbox{Var}(\tilde{X_n})\mbox{.}
\end{equation*}
Since the variance of $\bar{X_n}$ is lesser, it is the better estimator.
\pagebreak
\section*{Question 4 Solutions} % Unnumbered section
\textbf{Part I:}

a)
Null Hypothesis: $H_0\mbox{: }\mu = 5$

Alternative Hypothesis: $H_1\mbox{: }\mu \neq 5$

\textbf{NOTE:} For parts b) and c), we will use the rejection region method instead of the asymptotic test due to content restrictions.\\
b)

The rejection region method can be used here because we reject our null hypothesis, especially after the results show a different mean ($\mu = 6$), supporting the alternative hypothesis. The test statistic is
\[T_{\mu}(X)\coloneqq \left|\frac{\bar{X}-\mu}{\frac{\sigma}{n}}\right|\mbox{,}\]
\hspace*{6mm}and substituting our values into the statistic gives
\begin{align*}
    \left|\frac{6 - 5}{\frac{1.5}{100}}\right| = 6.67\mbox{ (to two decimal places).}
\end{align*}
c)

We set the rejection region to be
\begin{align*}
    R &= \left\{x; x\in\mathbb{R}^n\mbox{, }\left|\frac{\bar{X} - \mu_0}{\frac{\sigma}{\sqrt{n}}}\right| > c\right\}\\
    &= \left\{x; x\in\mathbb{R}^n\mbox{, }6.67 > c\right\}\mbox{.}
\end{align*}
We shall reject $H_0$ if $x\in R$ or equivalently $c < 6.67$.

Since the null hypothesis states that $\mu = 5 < 6.67$, we conclude that the average wait time is different from the target value of $5$ minutes.

\end{document}
